% Created 2018-09-18 Tue 13:23
% Intended LaTeX compiler: pdflatex
\documentclass[presentation]{beamer}
\usepackage[utf8]{inputenc}
\usepackage[T1]{fontenc}
\usepackage{graphicx}
\usepackage{grffile}
\usepackage{longtable}
\usepackage{wrapfig}
\usepackage{rotating}
\usepackage[normalem]{ulem}
\usepackage{amsmath}
\usepackage{textcomp}
\usepackage{amssymb}
\usepackage{capt-of}
\usepackage{hyperref}
\usepackage{amsthm, amssymb}
\usepackage{pgf,tikz,pgfplots}
\usepackage{graphicx}
\usepackage{colortbl}
\usepackage[french, frenchb]{babel}
\pgfplotsset{compat=1.13}
\usepgfplotslibrary{fillbetween}
\newtheorem{property}{Propriété}[section]
\newtheorem{defi}{Défi}[section]
\newtheorem{exe}{Exemple}[section]
\newtheorem{exo}{Exercice}[section]
\newtheorem{sol}{Solution}[section]
\newtheorem{rem}{Remarque}[section]
\newtheorem{demo}[theorem]{Démonstration}
\newcommand{\E}[1]{\ensuremath{\mathbb{#1}}}
\newcommand{\G}[3]{\ensuremath{(\E{#1}^{#2}, #3)}}
\newcommand{\M}[3]{\ensuremath{\left(\mathcal{M}_{#1}(\E{#2}), #3\right)}}
\newcommand{\tc}[2]{\ensuremath{\textcolor{#1}{#2}}}
\usetheme{default}
\usefonttheme{structurebold}
\useinnertheme{rectangles}
\useoutertheme{default}
\author{Laurent Garnier}
\date{}
\title{Cours de Terminale STMG : Information chiffrée}
\hypersetup{
pdfauthor={Laurent Garnier},
pdftitle={Cours de Terminale STMG : Information chiffrée},
pdfkeywords={},
pdfsubject={},
pdfcreator={Emacs 25.3.1 (Org mode 9.1.6)},
pdflang={Frenchb},
colorlinks,
citecolor=red,
filecolor=orange,
linkcolor=green,
urlcolor=magenta
}\begin{document}

\maketitle
\begin{frame}{Outline}
\tableofcontents
\end{frame}



\section{Rappels du programme officiel [1/3]}
\label{sec:orge7de06d}
\begin{frame}[label={sec:org2bc31e7}]{Rappels du programme [1/3]}
\begin{description}
\item[{Contenus}] Indice simple en base 100
\item[{Capacités attendues}] Passer de l'indice au taux d'évolution,
et réciproquement.
\item[{Commentaires}] Le calcul d'un indice synthétique, comme par
exemple l'indice des prix, n'est pas au programme.
\end{description}
\end{frame}

\begin{frame}[label={sec:org5f4f347}]{Indice simple en base 100 de y\(_{\text{2}}\) par rapport à y\(_{\text{1}}\)}
\begin{definition}
On appelle \alert{indice simple en base 100} de \(y_2\) par rapport à \(y_1\)
le nombre I tel que l'évolution qui fait passer de \(y_1\) à \(y_2\)
fait passer de 100 à I : \(I = 100\times k = 100\times \dfrac{y_2}{y_1}\)
Par commodité, on écrira seulement << indice de y\(_{\text{2}}\) par rapport à
y\(_{\text{1}}\) >>
\end{definition}

\begin{exe}
Un concessionnaire a vendu 800 voitures en janvier et 750 en
février. L'indice des ventes en février, base 100 en janvier, est
l'indice de \(y_2 = 750\) par rapport à \(y_1 = 800\), c'est-à-dire I =
100\texttimes{} \dfrac{750}{800} = 93,75
\end{exe}

\begin{defi}
En 2016 en France 32,5 millions de personnes consultaient YouTube
quotidiennement. En 2017 on est passé à 37,5. Calculer l'indice
base 100 en 2016.
\end{defi}
\end{frame}

\begin{frame}[label={sec:orgce3c1f2}]{Lien entre indice et taux d'évolution}
\begin{property}
L'indice I de y\(_{\text{2}}\) par rapport à y\(_{\text{1}}\) et le taux d'évolution t de y\(_{\text{1}}\)
à y\(_{\text{2}}\) sont reliés par les égalités :
\(I = 100\times (1 + t)\) et \(t = \dfrac{I-100}{100}\)
\end{property}

\begin{exe}
Une entreprise passe de l'indice I\(_{\text{1}}\) = 100 à l'indice I\(_{\text{2}}\) = 115. Calculons le
taux d'évolution : \(t = \dfrac{115 - 100}{100} = 0,15\)
\end{exe}

\begin{defi}
Quel est le taux d'évolution si on passe de l'indice I\(_{\text{1}}\) = 100 à
l'indice I\(_{\text{2}}\) = 97 ?
\end{defi}
\end{frame}
\section{Rappels du programme officiel [2/3]}
\label{sec:org03c50d0}
\begin{frame}[label={sec:org2b05f23}]{Rappels du programme [2/3]}
\begin{description}
\item[{Contenus}] Racine n-ième d'un réel positif. Notation \(a^{1/n}\)
\item[{Capacités attendues}] Déterminer avec une calculatrice ou un
tableur la solution positive de l'équation \(x^n = a\), lorsque
a est un réel positif.
\item[{Commentaires}] La notation \(\sqrt[n]{}\) n'est pas exigible.
\end{description}
\end{frame}

\begin{frame}[label={sec:org09adba9}]{Equations x\(^{\text{n}}\) = a, d'inconnue x dans l'intervalle [0;+\(\infty\)[}
\begin{definition}
On démontre que \alert{l'équation x\(^{\text{n}}\) = a} admet une unique solution dans
l'intervalle [0;+\(\infty\)[. Cette solution est notée a\(^{\text{1/n}}\)
\end{definition}

\begin{exe}
L'équation x\(^{\text{3}}\) = 8 admet une unique solution, x = 2.
\end{exe}

\begin{defi}
Quelle est l'unique solution de l'équation x\(^{\text{5}}\) = 32 ?
\end{defi}
\end{frame}

\begin{frame}[label={sec:org9700504}]{Racine n-ième d'un nombre réel positif ou nul}
\begin{definition}
On appelle \alert{racine n-ième de a} la solution a\(^{\text{1/n}}\) de l'équation
x\(^{\text{n}}\) = a dans l'intervalle [0; +\(\infty\)[.
\end{definition}

\begin{exe}
\begin{itemize}
\item 4 est la racine troisième de 64 parce que 4\(^{\text{3}}\) = 64
\item 3 est la racine quatrième de 81 parce que 3\(^{\text{4}}\) = 81
\item 2 est la racine septième de 64 parce que 2\(^{\text{7}}\) = 128
\end{itemize}
\end{exe}

\begin{defi}
\begin{itemize}
\item Trouver la racine troisième de 125.
\item Trouver la racine cinquième de 100 000.
\item Trouver la racine sixième de 64.
\end{itemize}
\end{defi}
\end{frame}

\section{Rappels du programme officiel [3/3]}
\label{sec:org41decc9}
\begin{frame}[label={sec:org5db3b5e}]{Rappels du programme [3/3]}
\begin{description}
\item[{Contenus}] Taux d'évolution moyen.
\item[{Capacités attendues}] Trouver le taux moyen connaissant le taux global.
\item[{Commentaires}] Exemple : taux mensuel équivalent à un taux annuel.
\end{description}
\end{frame}
\begin{frame}[label={sec:org59ea5cc}]{Taux d'évolution global}
\begin{definition}
On appelle \alert{taux d'évolution global} (ou \alert{taux global}) des n
évolutions successives, le taux d'évolution T de y\(_{\text{0}}\) à y\(_{\text{n}}\) :
\[1 + T = (1 + t_1)(1 + t_2)\dots (1 + t_n)\]
\end{definition}

\begin{exe}
Une entreprise connaît une hausse de 7\% entre 2015 et 2016 puis une
hausse de 4\% entre 2017 et 2018. Calculons le taux global.
1 + T = (1 + 0,07)(1 + 0,04) = 1,07\texttimes{} 1,04 = 1,1128 d'où T =
0,1128 = 11,28\%
\end{exe}

\begin{defi}
Une entreprise connaît une baisse de 7\% entre 2015 et 2016 puis une
baisse de 4\% entre 2017 et 2018. Calculer le taux global.
\end{defi}
\end{frame}
\begin{frame}[label={sec:orgc1e8a78}]{Taux d'évolution moyen}
\begin{definition}
On appelle \alert{taux d'évolution moyen} (ou \alert{taux moyen}) des n
évolutions successives, le nombre t\(_{\text{M}}\) tel que n évolutions
successives de même taux t\(_{\text{M}}\), partant de y\(_{\text{0}}\), aboutissent au même
nombre y\(_{\text{n}}\) que les évolutions précédentes. 
\((1+t_M)^n = 1 + T\) ainsi 1 + t\(_{\text{M}}\) est la racine n-ième de 1 + T.
\end{definition}

\begin{exe}
L'effectif d'un lycée a augmenté de 25\% en 3 ans. Calculons son
taux d'évolution moyen : (1 + t\(_{\text{M}}\))\(^{\text{3}}\) = 1,25 d'où 1 + t\(_{\text{M}}\) =
1,25\(^{\text{1/3}}\) \(\simeq\) 1,08 soit t\(_{\text{M}}\) \(\simeq\) 8\%
\end{exe}

\begin{defi}
Deux ans plus tard, le même lycée constate une augmentation de 18\%
en deux ans. Calculer le taux moyen.
\end{defi}
\end{frame}
\end{document}