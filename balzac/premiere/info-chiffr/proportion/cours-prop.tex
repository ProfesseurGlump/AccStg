% Created 2018-09-17 Mon 17:16
% Intended LaTeX compiler: pdflatex
\documentclass[presentation]{beamer}
\usepackage[utf8]{inputenc}
\usepackage[T1]{fontenc}
\usepackage{graphicx}
\usepackage{grffile}
\usepackage{longtable}
\usepackage{wrapfig}
\usepackage{rotating}
\usepackage[normalem]{ulem}
\usepackage{amsmath}
\usepackage{textcomp}
\usepackage{amssymb}
\usepackage{capt-of}
\usepackage{hyperref}
\usepackage{amsthm, amssymb}
\usepackage{pgf,tikz,pgfplots}
\usepackage{graphicx}
\usepackage{colortbl}
\usepackage[french, frenchb]{babel}
\pgfplotsset{compat=1.13}
\usepgfplotslibrary{fillbetween}
\newtheorem{property}{Propriété}[section]
\newtheorem{defi}{Défi}[section]
\newtheorem{exe}{Exemple}[section]
\newtheorem{exo}{Exercice}[section]
\newtheorem{sol}{Solution}[section]
\newtheorem{rem}{Remarque}[section]
\newtheorem{demo}[theorem]{Démonstration}
\newcommand{\E}[1]{\ensuremath{\mathbb{#1}}}
\newcommand{\G}[3]{\ensuremath{(\E{#1}^{#2}, #3)}}
\newcommand{\M}[3]{\ensuremath{\left(\mathcal{M}_{#1}(\E{#2}), #3\right)}}
\newcommand{\tc}[2]{\ensuremath{\textcolor{#1}{#2}}}
\usetheme{default}
\usefonttheme{structurebold}
\useinnertheme{rectangles}
\useoutertheme{default}
\author{Laurent Garnier}
\date{}
\title{Cours de 1\up{ère} STMG : Proportion}
\hypersetup{
pdfauthor={Laurent Garnier},
pdftitle={Cours de 1\up{ère} STMG : Proportion},
pdfkeywords={},
pdfsubject={},
pdfcreator={Emacs 25.3.1 (Org mode 9.1.6)},
pdflang={Frenchb},
colorlinks,
citecolor=red,
filecolor=orange,
linkcolor=green,
urlcolor=magenta
}
\begin{document}

\maketitle
\begin{frame}{Outline}
\tableofcontents
\end{frame}



\section{Rappels du programme officiel [1/3]}
\label{sec:org438787e}
\begin{frame}[label={sec:orgb6847e2}]{Rappels du programme [1/3]}
\begin{description}
\item[{Contenus}] Proportion d'une sous-population dans une
population.
\item[{Capacités attendues}] 
\end{description}


\begin{itemize}
\item Connaître et exploiter la relation entre effectifs et
proportion.
\item Associer proportion et pourcentage.
\end{itemize}
\begin{description}
\item[{Commentaires}] 
\end{description}


\begin{itemize}
\item Exemples : taux d'activité, taux de chômage, part de marché,
cote de popularité.
\item L'importance de la population de référence est soulignée.
\end{itemize}
\end{frame}

\begin{frame}[label={sec:org27a09e8}]{Institutionalisation [1/3]}
Dans toute cette partie, A et B désignent deux sous-populations
d'une population E.

\begin{definition}
La \alert{proportion de A dans E} est le nombre réel \(\dfrac{n_A}{n_E}\) où
\(n_A\) l'effectif de A et \(n_E\) l'effectif de E.
\end{definition}

\begin{exe}
Dans le monde 1,8 milliards de personnes se rendent sur YouTube chaque mois 
(satistique publiée en avril 2018). En France 37,5 millions de personnes s'y 
connectent chaque mois. La proportion des visiteurs français de YouTube est : 
p = \dfrac{37,5\times 10^6}{1,8\times 10^9} \(\simeq\) 2,1 \%
\end{exe}

\begin{defi}
Sachant que la population française est d'environ 65 millions
d'habitants, quelle est la proportion de visiteurs de YouTube en
France ?
\end{defi}
\end{frame}
\section{Rappels du programme officiel [2/3]}
\label{sec:orgfc644c3}
\begin{frame}[label={sec:org47faa7c}]{Rappels du programme [2/3]}
\begin{description}
\item[{Contenus}] Union et intersection de sous-populations.
\item[{Capacités attendues}] Pour deux sous-populations A et B d'une
population E, relier les proportions de A, de B, de \(A \cup B\)
et de \(A \cap B\).
\item[{Commentaires}] On peut étendre l'étude à plusieurs
sous-populations disjointes deux à deux ;
observer que pour une partition la somme des
fréquences vaut 1.
\end{description}
\end{frame}

\begin{frame}[label={sec:org61a02e0}]{Institutionalisation [2/3]}
\begin{definition}
L'\alert{intersection A \(\cap\) B} est la sous-population de E constituée
des individus appartenant à la fois \alert{à A et à B}. L'\alert{union A \(\cup\)
B} est la sous-population constituée des individus appartenant \alert{à A
ou à B}, c'est-à-dire ceux qui sont soit dans A, soit dans B, soit
dans les deux. 
\end{definition}

\begin{property}
\(p_A\), \(p_B\), \(p_{A\cap B}\), \(p_{A\cup B}\) désignent les
proportions associées. \(p_{A\cup B} = p_A + p_B - p_{A\cap B}\)
\end{property}

\begin{exe}
Soit E l'ensemble des 10 chiffres, A ceux inférieurs ou égaux à 5
et B ceux impairs. \(A\cap B\) = \{1, 3, 5\}, \(A\cup B\) = \{0, 1, 2,
3, 4, 5, 7, 9\} et \(p_{A\cup B} = 0,6 + 0,5 - 0,3 = 0,8\)
\end{exe}

\begin{defi}
On reprend la situation précédente avec B qui désigne les
pairs. Refaire tous les calculs.
\end{defi}
\end{frame}

\section{Rappels du programme officiel [3/3]}
\label{sec:orgc603c6b}
\begin{frame}[label={sec:org18faf33}]{Rappels du programme [3/3]}
\begin{description}
\item[{Contenus}] Inclusion
\item[{Capacités attendues}] 
\end{description}


\begin{itemize}
\item Connaître et exploiter la relation entre proportion de A
dans B, de B dans E et de A dans E, lorsque \(A\subset B\) et
\(B\subset E\).
\item Représenter des situations par des tableaux ou des arbres pondérés.
\end{itemize}
\begin{description}
\item[{Commentaires}] La notion de fréquence marginale est rencontrée
mais ce vocabulaire n'est pas exigible.
\end{description}
\end{frame}
\begin{frame}[label={sec:org03cd4ff}]{Institutionalisation [3/3]}
\begin{definition}
Lorsque tous les éléments d'un ensemble A appartiennent à un
ensemble B, on dit que \alert{A est inclus dans B} et on note \(A\subset B\).
\end{definition}

\begin{property}
Si \(A\subset B\) et \(B\subset E\), et si \(p_1\) est la proportion de A
dans B et \(p_2\) celle de B dans E alors celle de A dans E est \(p = p_1p_2\).
\end{property}

\begin{exe}
Dans une classe de 1\up{ère} STMG il y a 12 filles pour 18
garçons. Parmi les 12 filles 7 aiment faire du shopping.
La proportion des filles qui aiment faire du shopping dans cette
classe est : \(p = p_1p_2 = \dfrac{7}{12}\times \dfrac{12}{30} =
   \dfrac{7}{30} \simeq\) 23\%
\end{exe}

\begin{defi}
Dans cette même classe, parmi les 18 garçons, 12 aiment jouer au
foot. Quelle est la proportion des garçons de la classe qui aiment
jouer au foot ?
\end{defi}
\end{frame}
\end{document}