\chapter{Dossier documentaire}\etch{2}

\section{Analyses détaillées de travaux de recherches}

Le point d'ancrage du thème de ce mémoire est précisément le livre \cite{papert} et la note de lecture que nous en avons fait \cite{garnier3}.

\section{Analyses détaillées de documents pédagogiques}

Concernant les analyses détaillées de documents pédagogiques nous pouvons les regrouper en sous-sections.

\subsection{Analyse d'une séquence TICE}

Le sujet du TP1 sur la dérivation, la fiche professeur ainsi que l'analyse des séances sont consultables dans \cite{garnier2}.

\subsection{Analyse d'une évaluation <<traditionnelle>>}

Le sujet et l'analyse sont consultables dans \cite{garnier5}. Deux copies d'élèves sont consultables en annexes.

\section{Observations détaillées}

L'analyse détaillée d'une séance du tuteur de terrain est consultable dans \cite{garnier1}.