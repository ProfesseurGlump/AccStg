\documentclass{article}

\usepackage[a4paper, dvips]{geometry}
\usepackage{fontspec}
\usepackage{xunicode}
\usepackage{polyglossia}

\setmainlanguage{french}

\begin{document}

Bonjour à toutes et à tous,

 voici un questionnaire que j'ai réalisé afin d'essayer de découvrir
 la perception des mathématiques chez les élèves. 
Il n'y a aucune obligation à le faire, il n'y a ni bonne ni mauvaise réponse.
Le but est simplement d'essayer de comprendre comment vous percevez les mathématiques.
Répondre à ce questionnaire me permettra d'améliorer mes recherches
 sur cet aspect et peut être pourra améliorer la qualité de mon enseignement. 
Merci par avance à celles et ceux qui prendront la peine d'y répondre.

Il y a 10 questions pour ce questionnaire. \vspace{.25cm}

\newcounter{rep1}
\newenvironment{Phrase}{%
\begin{list}{Phrase \arabic{rep1} :}
{\usecounter{rep1}%
\setlength{\labelwidth}{1.8cm}%
\setlength{\labelsep}{0.1cm}%
\setlength{\leftmargin}{2cm}%
\setlength{\itemindent}{0cm}}}
{\end{list}}

\newcounter{rep2}
\newenvironment{Mot}{%
\begin{list}{Mot \arabic{rep2} :}
{\usecounter{rep2}%
\setlength{\labelwidth}{1.8cm}%
\setlength{\labelsep}{0.1cm}%
\setlength{\leftmargin}{2cm}%
\setlength{\itemindent}{0cm}}}
{\end{list}}

\newcounter{rep3}
\newenvironment{Mathématicien}{%
\begin{list}{Mathématicien \arabic{rep3} :}
{\usecounter{rep3}%
\setlength{\labelwidth}{1.8cm}%
\setlength{\labelsep}{0.1cm}%
\setlength{\leftmargin}{2cm}%
\setlength{\itemindent}{0cm}}}
{\end{list}}

\newcounter{rep5}
\newenvironment{Argument}{%
\begin{list}{Argument \arabic{rep5} :}
{\usecounter{rep5}%
\setlength{\labelwidth}{1.8cm}%
\setlength{\labelsep}{0.1cm}%
\setlength{\leftmargin}{2cm}%
\setlength{\itemindent}{0cm}}}
{\end{list}}

\newcounter{rep6}
\newenvironment{Qualité}{%
\begin{list}{Qualité \arabic{rep6} :}
{\usecounter{rep6}%
\setlength{\labelwidth}{1.8cm}%
\setlength{\labelsep}{0.1cm}%
\setlength{\leftmargin}{2cm}%
\setlength{\itemindent}{0cm}}}
{\end{list}}

\newcounter{rep8}
\newenvironment{Domaine}{%
\begin{list}{Domaine \arabic{rep8} :}
{\usecounter{rep8}%
\setlength{\labelwidth}{1.8cm}%
\setlength{\labelsep}{0.1cm}%
\setlength{\leftmargin}{2cm}%
\setlength{\itemindent}{0cm}}}
{\end{list}}

PERCEPTION DES MATHEMATIQUES.
  \begin{enumerate}
  \item Donner en quelques phrases (3 maximum) votre définition des
    mathématiques.
    \begin{Phrase}
      \item Pour moi les maths sont une suite de chiffres, de lettres
        qui forment des formules logique pour bute de trouver une
        données.   
      \item
      \item
    \end{Phrase}
  \item Donner une liste des 5 premiers mots qui vous viennent à l'esprit
    lorsque vous voyez ou entendez le mot <<mathématiques>>. 
    \begin{Mot}
      \item calculs
      \item équations
      \item formules
      \item torture
      \item 
    \end{Mot}
  \item Citez 4 noms de mathématiciens (inutile d'aller sur wikipédia
    le but est de savoir ceux que vous connaissez). 
    \begin{Mathématicien}
      \item 
      \item 
      \item 
      \item 
    \end{Mathématicien}
  \item Selon vous les hommes sont-ils plus aptes que les femmes pour
    faire des mathématiques ? 

    Je ne sais pas vraiment si les hommes sont plus disposés
    que les femmes à faire des maths mais je pense que beaucoup plus
    d'hommes que de femmes travaille dans le domaine des mathématiques. 
  \item Sans tenir compte des applications potentielles, selon vous est-il
    utile de faire des mathématiques ? Donnez 3 arguments pour justifier
    votre réponse.
    \begin{Argument}
      \item oui c'est utile car cela permet beaucoup d'avancées 
      \item seulement pour moi par exemple je suis convaincu que
        je ne ferais jamais de maths dans la vie future ou du moins pas
        des maths que nous faisons en cours alors il est vrais que je ne
        voie pas l'utilité.Combien de personne dans une classe ferra un
        métier en rapport avec vos maths ?
      \item 
    \end{Argument}
  \item Selon vous quelles sont les qualités requises pour faire des
    mathématiques ? Vous en donnerez 5 maximum. 
    \begin{Qualité}
      \item de la logique ! 
      \item il faut être attentif 
      \item et fournir du travaille 
      \item 
      \item 
    \end{Qualité}
  \item Selon vous quelles sont les qualités que la pratique
    mathématique permet de développer ? (C'est-à-dire des qualités que
    l'on avait peu ou pas et que les mathématiques nous ont permis
    d'acquérir ou de renforcer). Vous en donnerez 5 maximum. 
    \begin{Qualité}
      \item Les maths permette de renforcer l'esprit logique 
      \item elle permette aussi des avancées dans la science, enfin je pense 
      \item 
      \item 
      \item 
    \end{Qualité}
  \item Quels domaines associez-vous aux mathématiques (5 maximum) ?
    \begin{Domaine}
      \item la physique 
      \item et si vous parliez de métier, prof de maths ou mathématicien 
      \item 
      \item 
      \item 
    \end{Domaine}
  \item D'après vous quelle est la période historique durant laquelle le
    nombre de découvertes mathématiques a-t-il été le plus important ? 
    
    Je pense que le  XIXeme, XXeme siècle on joués un rôle important dans
    l'avancées des découvertes mathématique.
  \item Les mathématiques vous font-elles peur ?

    Oui, sa fait trois ans que j'ai décrochée en maths car j'ai eu la mal
    chance de tombée sur des professeurs assez médiocre, je suis larguée,
    je comprend sur le moment mais par la suite au contrôle je perd touts
    mes moyens et j'ai envie de travailler par ce que je sais très bien
    que les maths en ES c'est important mais je n'arrive pas à me mètre au
    travaille car rien que voir des maths aujourd'hui me démoralise. J'en
    suis arriver à un point ou je déteste vraiment les maths et même si
    j'ai l'envie de réussir je suis sur que je n'y arriverais pas. 
  \end{enumerate}


\end{document}
