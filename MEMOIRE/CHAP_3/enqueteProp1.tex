\documentclass{article}

\usepackage[a4paper, dvips]{geometry}
\usepackage{fontspec}
\usepackage{xunicode}
\usepackage{polyglossia}

\setmainlanguage{french}

\begin{document}

Bonjour à toutes et à tous,

 voici un questionnaire que j'ai réalisé afin d'essayer de découvrir
 la perception des mathématiques chez les élèves. 
Il n'y a aucune obligation à le faire, il n'y a ni bonne ni mauvaise réponse.
Le but est simplement d'essayer de comprendre comment vous percevez les mathématiques.
Répondre à ce questionnaire me permettra d'améliorer mes recherches
 sur cet aspect et peut être pourra améliorer la qualité de mon enseignement. 
Merci par avance à celles et ceux qui prendront la peine d'y répondre.

Il y a 10 questions pour ce questionnaire. \vspace{.25cm}

\newcounter{rep1}
\newenvironment{Phrase}{%
\begin{list}{Phrase \arabic{rep1} :}
{\usecounter{rep1}%
\setlength{\labelwidth}{1.8cm}%
\setlength{\labelsep}{0.1cm}%
\setlength{\leftmargin}{2cm}%
\setlength{\itemindent}{0cm}}}
{\end{list}}

\newcounter{rep2}
\newenvironment{Mot}{%
\begin{list}{Mot \arabic{rep2} :}
{\usecounter{rep2}%
\setlength{\labelwidth}{1.8cm}%
\setlength{\labelsep}{0.1cm}%
\setlength{\leftmargin}{2cm}%
\setlength{\itemindent}{0cm}}}
{\end{list}}

\newcounter{rep3}
\newenvironment{Mathématicien}{%
\begin{list}{Mathématicien \arabic{rep3} :}
{\usecounter{rep3}%
\setlength{\labelwidth}{1.8cm}%
\setlength{\labelsep}{0.1cm}%
\setlength{\leftmargin}{2cm}%
\setlength{\itemindent}{0cm}}}
{\end{list}}

\newcounter{rep5}
\newenvironment{Argument}{%
\begin{list}{Argument \arabic{rep5} :}
{\usecounter{rep5}%
\setlength{\labelwidth}{1.8cm}%
\setlength{\labelsep}{0.1cm}%
\setlength{\leftmargin}{2cm}%
\setlength{\itemindent}{0cm}}}
{\end{list}}

\newcounter{rep6}
\newenvironment{Qualité}{%
\begin{list}{Qualité \arabic{rep6} :}
{\usecounter{rep6}%
\setlength{\labelwidth}{1.8cm}%
\setlength{\labelsep}{0.1cm}%
\setlength{\leftmargin}{2cm}%
\setlength{\itemindent}{0cm}}}
{\end{list}}

\newcounter{rep8}
\newenvironment{Domaine}{%
\begin{list}{Domaine \arabic{rep8} :}
{\usecounter{rep8}%
\setlength{\labelwidth}{1.8cm}%
\setlength{\labelsep}{0.1cm}%
\setlength{\leftmargin}{2cm}%
\setlength{\itemindent}{0cm}}}
{\end{list}}

PERCEPTION DES MATHEMATIQUES.
  \begin{enumerate}
  \item Donner en quelques phrases (3 maximum) votre définition des
    mathématiques.
    \begin{Phrase}
      \item Les maths c'est un raisonnement logique, scientifique.
      \item les maths c'est des calculs.
      \item c'est duuur !
    \end{Phrase}
  \item Donner une liste des 5 premiers mots qui vous viennent à l'esprit
    lorsque vous voyez ou entendez le mot <<mathématiques>>. 
    \begin{Mot}
      \item calculs
      \item nombres
      \item logique
      \item scientifique
      \item einstein
    \end{Mot}
  \item Citez 4 noms de mathématiciens (inutile d'aller sur wikipédia
    le but est de savoir ceux que vous connaissez). 
    \begin{Mathématicien}
      \item einstein
      \item pythagore
      \item thales
      \item 
    \end{Mathématicien}
  \item Selon vous les hommes sont-ils plus aptes que les femmes pour
    faire des mathématiques ? 

    non !! tout le monde peut en faire 
  \item Sans tenir compte des applications potentielles, selon vous est-il
    utile de faire des mathématiques ? Donnez 3 arguments pour justifier
    votre réponse.
    \begin{Argument}
      \item oui mais jusqu a un certain niveau (stats,
        addition,multiplication) pour nous aider dans la vie de tous
        les jours
      \item non parce que les derivations n aide pas dans la vie
        quotidienne
      \item oui pour un meilleur raisonnement scientifique et pour les
        futurs metiers 
    \end{Argument}
  \item Selon vous quelles sont les qualités requises pour faire des
    mathématiques ? Vous en donnerez 5 maximum. 
    \begin{Qualité}
      \item mémoire
      \item bon raisonnement
      \item assiduité
      \item bonne compréhension
      \item serieux
    \end{Qualité}
  \item Selon vous quelles sont les qualités que la pratique
    mathématique permet de développer ? (C'est-à-dire des qualités que
    l'on avait peu ou pas et que les mathématiques nous ont permis
    d'acquérir ou de renforcer). Vous en donnerez 5 maximum. 
    \begin{Qualité}
      \item esprit plus logique
      \item medecine
      \item l'economie
      \item cuisine
      \item 
    \end{Qualité}
  \item Quels domaines associez-vous aux mathématiques (5 maximum) ?
    \begin{Domaine}
      \item médecine
      \item scientifique
      \item logique
      \item ingénieur
      \item intelligence
    \end{Domaine}
  \item D'après vous quelle est la période historique durant laquelle le
    nombre de découvertes mathématiques a-t-il été le plus important ? 
    
    je sais pas
  \item Les mathématiques vous font-elles peur ?

    oui beaucoup , trop difficile car trop de formule, trop de
    raisonnement incompréhensible, trop de chiffres  
  \end{enumerate}







\end{document}
