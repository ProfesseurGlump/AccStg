\documentclass[french]{report}


%%%%%%%%%%%%%%%%%%%%%%%%%%%%%%%%%%%%%%%%%%%%%%%%%%%%%%%%%%%%%%%%%%%%%%%%%%%%%%%%
%% paquetages ou extensions
%%%%%%%%%%%%%%%%%%%%%%%%%%%%%%%%%%%%%%%%%%%%%%%%%%%%%%%%%%%%%%%%%%%%%%%%%%%%%%%%

\usepackage[a4paper, dvips]{geometry}
\usepackage{fontspec}
\usepackage{xunicode}
\usepackage{polyglossia}

\setmainlanguage{french}
\setotherlanguage{english}

\usepackage{xltxtra}               % pour faire le logo XeLaTeX

\usepackage{mathtools, amssymb}          % pour les maths

%\usepackage{titlepage}
\usepackage{graphicx}
\usepackage{tabularx, longtable}
\usepackage{xspace}
\usepackage{verbatim}
\usepackage{shortvrb}
\usepackage{csquotes}

\usepackage{url}

\usepackage[xetex,breaklinks,colorlinks=true,%
        citecolor=blue,%
        urlcolor=blue,%
        pdftitle={Mémoire M2 MEEF},%
        pdfauthor={Laurent Garnier},%
        pdfsubject={Quelle approche pour soigner la mathophobie ?}]{hyperref}


% on appelle les paquetages chargées dans le ficihier
% paquetages.tex

%%%%%%%%%%%%%%%%%%%%%%%%%%%%%%%%%%%%%%%%%%%%%%%%%%%%%%%%%%%%%%%%%%%%%%%%%%%%%%%%

\author{Laurent \textsc{Garnier}}
\date{\today{}}


%%%%%%%%%%%%%%%%%%%%%%%%%%%%%%%%%%%%%%%%%%%%%%%%%%%%%%%%%%%%%%%%%%%%%%%%%%%%%%%%
% nouvelles commandes
% pour définir une liste de mots-clés avec la présentation comme abstract
\newenvironment{motcle}
        {\begin{center}\normalfont\bfseries Mots-clés
        \end{center}\begin{quote}}{\end{quote}\par}


%%%%%%%%%%%%%%%%%%%%%%%%%%%%%%%%%%%%%%%%%%%%%%%%%%%%%%%%%%%%%%%%%%%%%%%%%%%%%%%%

%%%%%%%%%%%%%%%%%%%%%%%%%%%%%%%%%%%%%%%%%%%%%%%%%%%%%%%%%%%%%%%%%%%%%%%%%%%%%%%%
% FIN DU PREAMBULE
%%%%%%%%%%%%%%%%%%%%%%%%%%%%%%%%%%%%%%%%%%%%%%%%%%%%%%%%%%%%%%%%%%%%%%%%%%%%%%%%

\begin{document}

\maketitle
\tableofcontents

\begin{abstract}
  Avant de rentrer dans le vif du sujet, l'auteur de ce document tient
  absolument à présenter ses excuses, par avance, à la gente féminine,
  pour le choix du genre masculin concernant le mot
  \textit{lecteur}. Ce choix est fait par simple soucis de commodité
  et non par sexisme. Nous sommes bien conscients que les lectrices
  peuvent être tout aussi perspicaces que les lecteurs. Ainsi vous
  serez priées mesdames, s'il vous plaît, de ne pas nous en tenir
  rigueur pour ce choix purement pratique d'économie de frappe du
  texte. Par conséquent à chaque utilisation du mot \textit{lecteur}, sachez
  bien que nous ferons référence en premier lieu aux lectrices de sexe
  féminin et en second lieu au lecteurs de sexe masculin. Mais
  concentrons-nous sur le propos du présent document.

  Ceci est une tentative, forcément infructueuse, de répondre à la
  question ainsi formulée en guise de titre. Tout d'abord il est
  important de bien comprendre le concept de \textit{mathophobie} et
  sa racine grecque <<mathêma>> (science) dérivant elle-même de
  <<manthanein>> (apprendre). La mathophobie désignera tantôt le
  problème lié à l'apprentissage des mathématiques en particulier,
  tantôt celui de l'apprentissage dans un cadre plus général; le
  contexte permettra de comprendre à quel sens réfèrera ce mot
  polysémique. La question de l'aspect psychologie de la peur ou de la
  méfiance ne sera pas abordée ici (pour plus de précisions sur ces
  aspects se référer à l'entrée \cite{siety} dans la
  bibliographie). Il sera question de l'apprentissage en général, de
  la nouveauté en particulier; tout cela dans un cadre mathématique orienté
  informatique. Naturellement, le temps imparti, la faiblesse des
  connaissances et l'expérience de l'auteur de ce mémoire, font que
  cela ne sera ébauche de réponse à cette question quasi-millénaire et
  dont personne ne détient de réponse unique. Puisse le lecteur averti
  nous pardonner cet ambitieux défi.
\end{abstract}

\begin{motcle}
  mathophobie, WIMS, motivation, confiance
\end{motcle}

{\rule{\linewidth}{1mm}}\vspace{.7mm}

J'autorise l'ESP\'E
\begin{itemize}
\item à exploiter le texte de mon mémoire dans la future formation des
  étudiants : \textsc{oui} à condition qu'il soit présenté sous format
  numérique garantissant ainsi la pleine exploitation du présent
  document (les liens ne sont pas très exploitable sous format
  papier).
\item à communiquer mon nom et mes coordonnées à de futurs étudiants
  MEEF qui souhaiteraient me contacter au sujet de mon mémoire :
  \textsc{oui} (sous réserve que la condition ci-dessus soit validée).
\end{itemize}

%\chapter{Partie centrale}\etch{1}

\input{CHAP_1/SECTIONS_1/intro}

\input{CHAP_1/SECTIONS_1/dev}

\input{CHAP_1/SECTIONS_1/conclusion}



\begin{thebibliography}{99}

\bibitem[(\textsc{Ascher}, 1998)]{ascher}
  \textsc{Ascher}, M., (1998),
  \textit{Mathématiques d'ailleurs Nombres, formes et jeux dans les
    sociétés traditionnelles (Ethnomatematics)},
  Paris : Seuil.

  Suite au cyclone qui a dévasté l'archipel du Vanuatu, un collègue
  professeur de français nous a fait suivre un lien pour faire un
  don. Et cela nous a rappelé que le livre de Marcia Ascher traite
  aussi des graphes eulériens appelés \textit{nitus} par les
  autochtones. Elle parle également d'application du groupe diédral
  pour comprendre les règles des mariages de certaines tribus
  aborigènes. Ce livre est une mine d'or pour qui veut sortir des
  sentiers battus en proposant des activités de découvertes
  ethnomathématiques. Attention toutefois, à part les graphes (qui
  concernent les options en terminale ES et S), les thèmes
  (mathématiques) évoqués sont de niveau universitaire et nécessite
  par conséquent un travail de simplification si l'on veut les
  présenter à des classes de lycée ou inférieur.

\bibitem[csu]{csu}
  \textsc{Canterbury}, Université, ($\leqslant 2009$),
  \textit{L'informatique sans ordinateur (Computer Science Unplugged)},
  \url{http://csunplugged.org}, dernière consultation avril 2015.
  Nouvelle-Zélande :
  \begin{english}
    University of Canterbury, NZ (aka ''Department of Fun Stuff'')
  \end{english}

  Il est très difficile de décrire les circonstances de la découverte
  fortuite de ce site. En effet, nous disposions d'un document titré
  \textit{Computer Science Unplugged} quelque part sur un disque
  dur\dots\xspace Quoiqu'il en soit, nous avons découvert ce site
  pendant les vacances et nous y avons trouvé de nombreuses inspirations
  d'activités introduisant les concepts informatiques sans ordinateurs
  (numération binaire, recherche dichotomique, initiation à la
  cryptographie\dots). On trouvera sur le site de nombreuses vidéos
  (une seule a été traduite en français) illustrant clairement la mise
  en place de ce type d'activité dans des écoles primaires. Nous n'avons
  pas encore eu l'occasion de les mettre en \oe{}uvre avec nos
  classes. Néanmoins nous espérons pouvoir le faire à titre d'activités
  ludiques de fin d'année. Il est à noter que cela pourrait être
  intéressant de proposer à ceux qui font les programmes de lycée (ou
  de collège) d'utiliser ce genre de méthodes afin d'introduire les
  mathématiques discrètes (on pourra consulter par exemple
  \url{https://interstices.info/jcms/c_47072/enseigner-et-apprendre-les-sciences-informatiques-a-lecole}
  pour voir qu'il s'agit d'un point de vue partagé par la communauté
  de la recherche (en informatique)). Le lien fourni ci-avant invite à
  une entrée plus détaillée, plus bas dans la biblio/sito-graphie.

\bibitem[(\textsc{\'Eveilleau}, 2014)]{eveilleau}
  \textsc{\'Eveilleau}, T., ($\leqslant 2001$?),
  \textit{Mathématiques magiques},
  \url{http://therese.eveilleau.pagesperso-orange.fr/}, consulté à
  partir de décembre 2014.

  Lors d'un travail de recherche sur les solides de Platon, l'un de
  nos élèves de seconde à trouvé (par hasard ?) des animations des
  patrons de ces solides sur le site de Thérèse
  \'Eveilleau. Saisissant l'occasion d'encourager les prises
  d'initiatives, nous avons recommandé aux autres élèves de consulter ce
  site. Puis, nous avons découvert à notre grande surprise la richesse des
  thèmes évoqués. D'ailleurs, nous en avons profité pour saluer la
  conceptrice (ex-formatrice IUFM) par un courriel de
  remerciements. Plus tard dans l'année lors d'une journée de
  formation madame \textsc{Lallier-Girot} nous a fourni un document
  extrait de ce site. Nous ne saurions vous recommander suffisamment
  la consultation de ce site (tours de magies, modélisation 3D,
  devinettes logiques\dots). Attention, si vous souhaitez mettre en
  \oe{}uvre les tours de magies sans ordinateur, il vous faudra
  découvrir par vous même les <<trucs>> (ce qui donne déjà une
  satisfaction pour l'enseignant avant même de les présenter aux aprenants).

\bibitem[(\textsc{Garnier}, 2014, visiteTuteur.pdf)]{garnier1}
  \textsc{Garnier}, L., (2014),
  \url{http://eprel.u-pec.fr/eprel/claroline/backends/download.php?url=L0dST1VQRV80X19fVGhvbWFzLF9MYXVyZW50LF9JbWVuL0dBUk5JRVJkb2NzL3Zpc2l0ZVR1dGV1ci5wZGY\%3D\&currentTime=1431551935\&cidReset=true\&cidReq=7223},
  déposée sur Eprel le \date{13 mai 2015}

  Il s'agit d'une analyse de l'observation d'une séance du tuteur de
  terrain. Cette visite s'est déroulée au début du mois de novembre 2014.

\bibitem[(\textsc{Garnier}, 2014, tp1Deriv.pdf)]{garnier2}
  \textsc{Garnier}, L., (2014),
  \url{http://eprel.u-pec.fr/eprel/claroline/document/document.php?cmd=exChDir\&file=L0dST1VQRV80X19fVGhvbWFzLF9MYXVyZW50LF9JbWVuL0dBUk5JRVJkb2NzL1RQMXByZW1FUw\%3D÷\%3D\&cidReset=true\&cidReq=7223},
  déposée sur Eprel le \date{13 mai 2015}

  Il s'agit d'un premier TP sur geogebra organisé sur 4 séances (la
  dernière a été évaluative) d'une heure chacune (sur 4 semaines) en
  parallèle du cours sur la dérivation. Le thème fondamental est la
  notion de taux de variation, nombre dérivé et tangente en un
  point. La classe cible est une classe de première filière ES.

\bibitem[(\textsc{Garnier}, 2015, GarnierNot2Lect.pdf)]{garnier3}
  \textsc{Garnier}, L., (2015),
  \url{http://eprel.u-pec.fr/eprel/claroline/backends/download.php?url=L0dST1VQRV80X19fVGhvbWFzLF9MYXVyZW50LF9JbWVuL0dBUk5JRVJkb2NzL0dhcm5pZXJOb3QyTGVjdC5wZGY\%3D\&currentTime=1431553745\&cidReset=true\&cidReq=7223},
  déposée sur Eprel le \date{13 mai 2015}


\bibitem[(\textsc{Garnier}, 2015, TP2Deriv.pdf)]{garnier4}
  \textsc{Garnier}, L., (2015),
  \url{http://eprel.u-pec.fr/eprel/claroline/document/document.php?cmd=exChDir\&file=L0dST1VQRV80X19fVGhvbWFzLF9MYXVyZW50LF9JbWVuL0dBUk5JRVJhbm5leGVzL1RQMnByZW1FUw\%3D\%3D\&cidReset=true\&cidReq=7223},
  déposée sur Eprel le \date{13 mai 2015}

  Il s'agit du second TP sur geogebra organisé sur 3 séances (la dernière
  séance a été évaluative) d'une heure chacune (sur 4 semaines 2 avant
  les vacances et 2 pendant les vacances) en parallèle du cours sur
  les études de fonctions. Le thème fondamental est la notion de
  fonction dérivée et variation d'une fonction. La classe cible est une
  classe de première filière ES.


\bibitem[(\textsc{Gazagnes}, 2015)]{gazagnes}
  \textsc{Gazagnes}, A., ($\leqslant 2012$?),
  \textit{\LaTeX{}\dots\xspace pour le prof de maths !},
  \url{http://math.univ-lyon1.fr/irem/IMG/pdf/LatexPourProfMaths.pdf},
  dernière consultation le \today{}.

  La rédaction des différentes notes de lecture, des cours pour les
  élèves ou tout autre document structuré est grandement facilitée et
  améliorée par l'usage des langages \TeX{} (\LaTeX{} ou \XeTeX{}). Le
  présent guide est rédigé par un membre de l'IREM de Lyon (gage de
  qualité et d'uilité de l'information qui y est délivrée). Il permet
  non seulement d'obtenir des idées concernant la forme des documents
  que l'on produit mais par delà la forme, il permet d'opérer plus
  finement la distinction entre fond et forme. Ainsi, le découpage des
  idées et la rédaction de documents sous formes de structures
  arborescentes et également hypertextuelles (dans le second cas on
  peut clairement parler de structure de graphes qui confère à ce type
  de document la possibilité d'y faire des liens internes
  (auto-référents) et externes (url)) permettent de développer
  davantage l'aspect transformable (recyclable) d'un document
  polymorphe (insertion de graphique, d'image, de lien url, vidéo
  \dots) via des combinaisons, modifications et améliorations succesives.

\bibitem[interstices]{interstices}
  \textsc{Interstices}, ($\leqslant 2007$?),
  \textit{Explorez les sciences du numérique},
  \url{https://interstices.info/jcms/jalios_5127/accueil}, dernière
  consultation le \today{}.

  Il n'y a pas grand chose à dire de plus que ce qui a été dit plus
  haut concernant l'ensemble d'activités consultables sur
  \cite{csu}. Néanmoins, le lien proposé ici est celui de la page
  d'accueil qui offre l'avantage de trouver de nombreux articles sur
  l'informatique, les mathématiques et l'éducation. Par exemple on y
  trouvera des thèmes comme la récursivité (récurrence) avec les Tours
  de Hanoï par exemple mais également des réflexions sur les équations
  différentielles. Il y a beaucoup d'articles écrit depuis au moins
  2007. Et l'avantage est qu'ils sont tous rédigés en français.

\bibitem[(\textsc{Papert}, 1981)]{papert}
  \textsc{Pappert}, S., (1981),
  \textit{Jaillissement de l'esprit, ordinateurs et apprentissage
    (Mindstorms children, computers, and powerful ideas)},
  Paris : Champs Flammarion.

  L'enthousiasme de Seymour Pappert pour l'utilisation intelligente de
  l'ordinateur pour améliorer l'apprentissage est contagieux. \`A lire
  absolument afin d'acquérir une vraie réflexion sur les raisons de
  l'utilité de la machine. Bien que daté de plus de 30 ans les
  réflexions développées sont encore d'actualité. Un chapitre (le
  deuxième) est notamment consacré à la mathophobie (pour plus de précision se
  référer à \cite{garnier3} pour une note de lecture sur le sujet).

\bibitem[(\textsc{Siéty}, 2012)]{siety}
  \textsc{Siéty}, A., (2012),
  \textit{Qui a peur des mathématiques ?},
  Paris : Denoël.

  Approche originale mêlant véritables problèmes mathématiques avec
  une vision psychologique des traumatismes des apprenants. Ce livre
  m'a permis de rencontrer Anne Siéty de façon informelle mais très
  enrichissante. De nombreuses anecdotes permettent d'avoir un regard
  neuf sur les erreurs de nos élèves.

\bibitem[(\textsc{Stewart}, 2013)]{stewart}
  \textsc{Stewart}, I., (2013),
  \textit{Mon cabinet de curiosités mathématiques (Professor Stewart's
  Cabinet of Mathematical Curiosities)},
  Paris : Flammarion.

  Ian Stewart est un passionné de curiosité mathématiques. On trouvera
  dans ce livre de nombreuses idées d'activités et autres devinettes
  ou colles pour les élèves ou pour soi-même (peut même se lire dans
  le métro). Probabilité, géométrie,
  topologie, arithmétique\dots\xspace Tous les thèmes sont passés en revue.



\end{thebibliography}


%\section{Note de lecture}
\documentclass[11pt, french]{article}

\input{/Users/professeurglump/Desktop/ProgLycee/TeX/MacroTeX/pkg.tex}


\title{Note de lecture M2MEEF maths}
\author{Laurent Garnier}
\date{\today}


\input{/Users/professeurglump/Desktop/ProgLycee/TeX/MacroTeX/cmdLycee.tex}

 %%%%%%%%%%%%%%%%%%%%%%%%%%%%%%%%%%%%%%%%%
 \begin{document}

 \maketitle

 \tableofcontents
 


\input{bodyNote2Lect.tex}


\end{document}


%\appendix

\chapter{Analyse d'évaluation}

Dans cette partie vous pourrez trouver une analyse d'évaluation qui
s'est déroulée en classe de seconde sur le thème des
probabilités.

Vous trouverez :
\begin{itemize}
\item Le sujet du devoir sur table\footnote{Consulter le lien
    \url{http://eprel.u-pec.fr/eprel/claroline/backends/download.php?url=L0dST1VQRV80X19fVGhvbWFzLF9MYXVyZW50LF9JbWVuL0dBUk5JRVJhbm5leGVzL0FuYWx5c2VFdmFsL0RTcHJvYi5wZGY\%3D\&currentTime=1431555968\&cidReset=true\&cidReq=7223}
    déposé sur EPREL le \date{13 mai 2015}}.
\item Un exemple de <<\textit{bonne}>> copie\footnote{Consulter le
    lien \url{http://eprel.u-pec.fr/eprel/claroline/backends/download.php?url=L0dST1VQRV80X19fVGhvbWFzLF9MYXVyZW50LF9JbWVuL0dBUk5JRVJhbm5leGVzL0FuYWx5c2VFdmFsL0JDLnBkZg\%3D\%3D\&currentTime=1431556141\&cidReset=true\&cidReq=7223} déposé sur sur EPREL le \date{13 mai 2015}}.
\item Un exemple de <<\textit{mauvaise}>> copie\footnote{Consulter le
    lien \url{http://eprel.u-pec.fr/eprel/claroline/backends/download.php?url=L0dST1VQRV80X19fVGhvbWFzLF9MYXVyZW50LF9JbWVuL0dBUk5JRVJhbm5leGVzL0FuYWx5c2VFdmFsL01DLnBkZg\%3D\%3D\&currentTime=1431556211\&cidReset=true\&cidReq=7223} déposé sur sur EPREL le \date{13 mai 2015}}.
\item Ainsi que l'évaluation à proprement parler\footnote{Consulter le
    lien \url{http://eprel.u-pec.fr/eprel/claroline/backends/download.php?url=L0dST1VQRV80X19fVGhvbWFzLF9MYXVyZW50LF9JbWVuL0dBUk5JRVJhbm5leGVzL0FuYWx5c2VFdmFsL2ZpY2hlUHJvZi5wZGY\%3D\&currentTime=1431556266\&cidReset=true\&cidReq=7223} déposé sur sur EPREL le \date{13 mai 2015}}.
\end{itemize}





\end{document}
