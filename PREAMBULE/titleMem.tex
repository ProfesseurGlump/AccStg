\newlength{\larg}
\setlength{\larg}{\linewidth}


\title{
\begin{tabular}{lr}
\begin{minipage}{.25\linewidth}
\includegraphics[scale=.35]{../PICS/espe}
\end{minipage}
&
\begin{minipage}{.25\linewidth}
\includegraphics[scale=.35]{../PICS/upec}
\end{minipage}
\end{tabular}\\
{\rule{\larg}{1mm}}\vspace{7mm}
\begin{center}
\begin{tabular}{ll}
  {\sc Académie : } & {\sc Créteil} \\
  {\sc Master : }   & {\sc MEEF Second degré} \\
  {\sc parcours : } & {\sc Mathématiques}\\
  {\sc Année : }    & {\sc 2014/2015}\\        
\end{tabular}
\end{center}
{\rule{\larg}{1mm}}\vspace{.7mm}
\begin{minipage}{.75\linewidth}
  \begin{center}
  {\huge Titre : \bf Quelle approche pour soigner la
  mathophobie}\footnote{Ce néologisme est dû à S. \textsc{Pappert},
  pour plus de détails consulter \cite[Chapitre 2, p.53]{papert}}? 
  \end{center}
  \end{minipage}
{\rule{\larg}{1mm}}\vspace{.7mm}
\begin{center}
\begin{tabular}{c}                
  \begin{minipage}{.75\linewidth}
  \begin{center}
  {\sc \'Etablissement de stage : lycée polyvalent Adolphe Chérioux}
  \end{center}
  \end{minipage}\\
  \\
  {\sc Responsable de suivi de mémoire :  Sylviane Schwer}
\end{tabular}
\end{center}
}
\author{Laurent \textsc{Garnier}}
\date{\today{}}
